\section{结论}
本文结合简历系统信息数据以产生对文化程度,工种,驻地的相关性预测。针对整个数据集,本文中所描述用户信息数据存在稀疏性问题,对于此问题,本文中采用对用户信息数据进行特征选择后以地理位置作为初始分类原则,将用户归类,然后对某一类用户进行聚类分析(这一过程本文中是通过对已处理后的数据属性来进行的)。之后,所产生的用户集具有统一分析的参考价值。

接下来,本文中将对录入信息也做了一些预处理,进行了异常项检测,并对异常项进行了处理,以保障进入决策树的数据集是不包含脏数据的可以直接进行分类研究的可靠数据。

然后,本文中构造两个不同的决策树以对不同关联lable的相似度进行分别研究,探寻驻地,工种,文化程度之间的相关性。

最后,本文中通过Aplly Model对不同文化程度对应工种,对应驻地进行了预测值分布预测,以及对不同驻地对应不同工种进行了预测值分布预测,揭示了海量数据的数据价值。
